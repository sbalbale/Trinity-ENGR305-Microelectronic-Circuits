\documentclass[11pt]{article}
\usepackage[margin=1in]{geometry}
\usepackage{amsmath}
\usepackage{siunitx} % For proper unit formatting
\usepackage{booktabs} % For tables
\usepackage{circuitikz} % For circuit drawings
\usepackage{float} % For [H] figure placement

% Set paragraph spacing instead of indentation
\setlength{\parindent}{0in}
\setlength{\parskip}{1ex}

\title{ENGR 305 Lab 10: Hand Calculations}
\author{Sean Balbale}
\date{\today}

\begin{document}

\maketitle
\thispagestyle{empty}

\section{Part 1: Design and DC Analysis}

\subsection{Given Parameters}
\begin{itemize}
  \item \textbf{Supplies:} $V_{+} = \SI{15}{\volt}$, $V_{-} = \SI{-15}{\volt}$
  \item \textbf{Design Goals:} $I_{C} = \SI{1}{\milli\ampere}$, Gain $A_v = -200$ V/V
  \item \textbf{Components:} $R_{sig} = \SI{50}{\ohm}$, $R_{L} = \SI{10}{\kilo\ohm}$, $R_{B} = \SI{10}{\kilo\ohm}$
  \item \textbf{Transistor Model:} 2N3904 NPN, assume $\beta = 100$, $V_{BE} = \SI{0.7}{\volt}$, $V_T \approx \SI{26}{\milli\volt}$
\end{itemize}

\subsection{DC Operating Point Analysis}
For the DC analysis, all coupling capacitors ($C_{c1}, C_{c2}$) and bypass capacitors ($C_E$) are treated as open circuits. The base is connected to ground through $R_B$.

\begin{figure}[H]
  \centering
  \begin{circuitikz}
    % Rails
    \node[vcc](Vcc) at (2, 3) {$V_+ = \SI{15}{\volt}$};
    \node[vee](Vee) at (2, -3) {$V_- = \SI{-15}{\volt}$};

    % Transistor
    \node[npn](Q1) at (2, 0) {};

    % Resistors
    \draw (Q1.C) to[R, l=$R_C$] (Q1.C |- Vcc.south) -- (Vcc);
    \draw (Q1.E) to[R, l=$R_E$] (Q1.E |- Vee.north) -- (Vee);

    % Base Resistor to Ground
    \draw (Q1.B) -- ++(-1,0) coordinate(Bnode);
    \draw (Bnode) to[R, l=$R_B$, a=$\SI{10}{k\ohm}$] ++(0,-2) node[ground]{};

    % Labels
    \draw (Q1.B) node[above left] {$V_B$};
    \draw (Q1.E) node[right] {$V_E$};
    \draw (Q1.C) node[right] {$V_C$};
  \end{circuitikz}
  \caption{DC Equivalent Circuit}
\end{figure}

\textbf{1. Calculate Base and Emitter Currents ($I_B, I_E$)} \\
Using the target collector current $I_C = \SI{1}{\milli\ampere}$ and $\beta = 100$:
$$
I_B = \frac{I_C}{\beta} = \frac{\SI{1}{\milli\ampere}}{100} = \SI{10}{\micro\ampere}
$$
$$
I_E = I_C + I_B = \SI{1}{\milli\ampere} + \SI{0.01}{\milli\ampere} = \SI{1.01}{\milli\ampere}
$$

\textbf{2. Calculate Base and Emitter Voltages ($V_B, V_E$)} \\
The base current flows from ground, through $R_B$, into the base.
$$
V_B = 0 - I_B R_B = 0 - (\SI{10}{\micro\ampere})(\SI{10}{\kilo\ohm}) = \SI{-0.1}{\volt}
$$
Using the standard assumption of a \SI{0.7}{\volt} drop across the base-emitter junction:
$$
V_E = V_B - V_{BE} = \SI{-0.1}{\volt} - \SI{0.7}{\volt} = \SI{-0.8}{\volt}
$$

\textbf{3. Design Emitter Resistor ($R_E$)} \\
$R_E$ sets the bias current. It is calculated from the voltage drop across it ($V_E - V_-$) and $I_E$.
$$
R_E = \frac{V_E - V_-}{I_E} = \frac{\SI{-0.8}{\volt} - (\SI{-15}{\volt})}{\SI{1.01}{\milli\ampere}} = \frac{\SI{14.2}{\volt}}{\SI{1.01}{\milli\ampere}} \approx \SI{14.06}{\kilo\ohm}
$$
\textit{Practical Note:} $\SI{14.06}{\kilo\ohm}$ is not a standard E12/E24 value. It can be approximated using a decade box or a series combination (e.g., $\SI{10}{\kilo\ohm} + \SI{3.9}{\kilo\ohm} + \SI{160}{\ohm}$).

\section{Part 2: AC Small-Signal Analysis}

\subsection{Small-Signal Parameters}
First, we calculate the transconductance ($g_m$) and input resistance ($r_{\pi}$) using the thermal voltage $V_T \approx \SI{26}{\milli\volt}$:
$$
g_m = \frac{I_C}{V_T} = \frac{\SI{1}{\milli\ampere}}{\SI{26}{\milli\volt}} \approx \SI{38.46}{\milli\siemens}
$$
$$
r_{\pi} = \frac{\beta}{g_m} = \frac{100}{\SI{38.46}{\milli\siemens}} \approx \SI{2.6}{\kilo\ohm}
$$

\subsection{Circuit Model}
In the small-signal analysis, DC supplies become AC grounds, and large capacitors act as short circuits. The emitter resistor $R_E$ is bypassed by $C_E$, connecting the emitter directly to ground.

\begin{figure}[H]
  \centering
  \begin{circuitikz}[american]
    % Input
    \draw (0,0) node[ground]{} to[R, l=$R_B$] (0,2) -- (2,2) node[label=above:$v_i$]{};

    % Hybrid-pi model
    \draw (2,2) to[R, l=$r_{\pi}$, v=$v_{\pi}$] (2,0) node[ground]{};
    \draw (4,0) node[ground]{} to[cI, l=$g_m v_{\pi}$] (4,2) -- (6,2);

    % Output side
    \draw (6,2) to[R, l=$R_C$] (6,0) node[ground]{};
    \draw (6,2) -- (8,2) to[R, l=$R_L$] (8,0) node[ground]{};
    \draw (8,2) -- (9,2) node[right] {$v_o$};

    % Connections
    \draw (2,2) -- (4,2);
  \end{circuitikz}
  \caption{Small-Signal Model (Hybrid-$\pi$)}
\end{figure}

\subsection{Gain Derivation and Design}
\textbf{1. Voltage Gain Expression ($A_v$)} \\
The output voltage is generated by the dependent current source pulling current through the parallel combination of $R_C$ and $R_L$:
$$
v_o = -(g_m v_{\pi}) (R_C || R_L)
$$
Since $v_{\pi} = v_i$ (emitter is grounded), the gain $A_v$ is:
$$
A_v = \frac{v_o}{v_i} = -g_m (R_C || R_L)
$$

\textbf{2. Design Collector Resistor ($R_C$)} \\
We require a gain of $A_v = -200$.
$$
|-200| = g_m (R_C || R_L) \implies 200 = (\SI{38.46}{\milli\siemens}) \left( \frac{R_C R_L}{R_C + R_L} \right)
$$
Solving for the equivalent resistance $R_{eq} = R_C || R_L$:
$$
R_{eq} = \frac{200}{\SI{38.46}{\milli\siemens}} \approx \SI{5.2}{\kilo\ohm}
$$
Now solve for $R_C$ knowing $R_L = \SI{10}{\kilo\ohm}$:
$$
\frac{1}{R_{eq}} = \frac{1}{R_C} + \frac{1}{R_L} \implies \frac{1}{R_C} = \frac{1}{\SI{5.2}{\kilo\ohm}} - \frac{1}{\SI{10}{\kilo\ohm}}
$$
$$
\frac{1}{R_C} \approx 1.923 \times 10^{-4} - 1 \times 10^{-4} = 0.923 \times 10^{-4} \text{ S}
$$
$$
R_C \approx \SI{10.83}{\kilo\ohm}
$$
\textit{Practical Note:} This can be implemented with a $\SI{10}{\kilo\ohm}$ and an $\SI{820}{\ohm}$ resistor in series, or using a decade box.

\subsection{Verification and Analysis}
\textbf{1. DC Collector Voltage ($V_C$)} \\
We must verify the transistor remains in the active region ($V_C > V_B$).
$$
V_C = V_+ - I_C R_C = \SI{15}{\volt} - (\SI{1}{\milli\ampere})(\SI{10.83}{\kilo\ohm}) = \SI{15}{\volt} - \SI{10.83}{\volt} = \SI{4.17}{\volt}
$$
Since $V_C (\SI{4.17}{\volt}) > V_B (\SI{-0.1}{\volt})$, the transistor is in the Active Region.

\textbf{2. Output Resistance ($R_o$)} \\
The output resistance looking into the collector is approximately $R_C$ (ignoring the transistor's early voltage $V_A$).
$$
R_o \approx R_C = \SI{10.83}{\kilo\ohm}
$$

\textbf{3. Input Attenuation ($v_i / v_{sig}$)} \\
The input impedance $Z_{in}$ looking into the base is $R_B || r_{\pi}$:
$$
Z_{in} = \frac{R_B r_{\pi}}{R_B + r_{\pi}} = \frac{10 \cdot 2.6}{10 + 2.6} \si{k\ohm} \approx \SI{2.06}{\kilo\ohm}
$$
The signal reaching the base $v_i$ is a voltage divider with $R_{sig} = \SI{50}{\ohm}$:
$$
\frac{v_i}{v_{sig}} = \frac{Z_{in}}{Z_{in} + R_{sig}} = \frac{\SI{2060}{\ohm}}{\SI{2060}{\ohm} + \SI{50}{\ohm}} \approx 0.976
$$
This is close to unity, so the overall system gain is approximately equal to the stage gain $A_v$.

\section{Summary of Values}
The following table summarizes all assumed, given, and calculated values used in this lab analysis.

\begin{table}[H]
  \centering
  \caption{Summary of Circuit Parameters}
  \label{tab:summary}
  \renewcommand{\arraystretch}{1.3}
  \begin{tabular}{@{}l l r@{}}
    \toprule
    \textbf{Parameter} & \textbf{Symbol} & \textbf{Value} \\
    \midrule
    \multicolumn{3}{c}{\textit{Given \& Assumed Constants}} \\
    Positive Supply Voltage & $V_+$ & \SI{15}{\volt} \\
    Negative Supply Voltage & $V_-$ & \SI{-15}{\volt} \\
    Target Collector Current & $I_C$ & \SI{1}{\milli\ampere} \\
    Transistor Beta & $\beta$ & 100 \\
    Base-Emitter Voltage & $V_{BE}$ & \SI{0.7}{\volt} \\
    Thermal Voltage & $V_T$ & \SI{26}{\milli\volt} \\
    Signal Source Resistance & $R_{sig}$ & \SI{50}{\ohm} \\
    Load Resistance & $R_L$ & \SI{10}{\kilo\ohm} \\
    Base Bias Resistor & $R_B$ & \SI{10}{\kilo\ohm} \\
    Target Gain & $A_v$ & -200 V/V \\
    \midrule
    \multicolumn{3}{c}{\textit{Calculated DC Values}} \\
    Base Current & $I_B$ & \SI{10}{\micro\ampere} \\
    Emitter Current & $I_E$ & \SI{1.01}{\milli\ampere} \\
    Base Voltage & $V_B$ & \SI{-0.1}{\volt} \\
    Emitter Voltage & $V_E$ & \SI{-0.8}{\volt} \\
    Emitter Resistor (Calculated) & $R_E$ & \SI{14.06}{\kilo\ohm} \\
    Collector Resistor (Calculated) & $R_C$ & \SI{10.83}{\kilo\ohm} \\
    Collector Voltage & $V_C$ & \SI{4.17}{\volt} \\
    \midrule
    \multicolumn{3}{c}{\textit{Calculated AC Values}} \\
    Transconductance & $g_m$ & \SI{38.46}{\milli\siemens} \\
    Input Resistance (Base) & $r_{\pi}$ & \SI{2.6}{\kilo\ohm} \\
    Input Impedance (Total) & $Z_{in}$ & \SI{2.06}{\kilo\ohm} \\
    Output Resistance & $R_o$ & \SI{10.83}{\kilo\ohm} \\
    Input Attenuation Ratio & $v_i / v_{sig}$ & 0.976 \\
    \bottomrule
  \end{tabular}
\end{table}

\end{document}
