\documentclass[11pt]{article}
\usepackage[margin=1in]{geometry}
\usepackage{siunitx}
\usepackage{circuitikz}
\usepackage{float}

% Remove paragraph indentation
\setlength{\parindent}{0in}

\begin{document}

% --- Figure 1: Full Experimental Schematic ---
\begin{circuitikz}[american]
  % Power Rails
  \node[vcc](Vcc) at (4, 5) {$V_+ = \SI{15}{V}$};
  \node[vee](Vee) at (4, -5) {$V_- = \SI{-15}{V}$};

  % Transistor
  \node[npn](Q1) at (4, 0) {2N3904};

  % Bias Resistors
  \draw (Q1.C) to[R, l=$R_C$] (Q1.C |- Vcc.south) -- (Vcc);
  \draw (Q1.E) to[R, l=$R_E$] (Q1.E |- Vee.south) -- (Vee);
  \draw (Q1.B) -- ++(-1,0) coordinate(Bnode);
  \draw (Bnode) to[R, l=$R_B$] (Bnode |- Vee.north) node[ground]{}; % RB to Ground

  % Input Stage (AC Coupled)
  \draw (Bnode) to[C, l=$C_{c1}$, -o] ++(-2.5, 0) coordinate(Vin);
  \draw (Vin) to[R, l=$R_{sig}$] ++(-2, 0) coordinate(Vsource);
  \draw (Vsource) to[sV, l=$v_{sig}$] (Vsource |- Vee.north) node[ground]{};

  % Output Stage (AC Coupled)
  \draw (Q1.C) to[C, l=$C_{c2}$, -o] ++(2.5, 0) coordinate(Vout);
  \draw (Vout) to[R, l=$R_L$] (Vout |- Vee.north) node[ground]{};
  \draw (Vout) -- ++(0.5, 0) node[right] {$v_o$};

  % Emitter Bypass
  % Note: CE connects Emitter to Ground to bypass RE
  \draw (Q1.E) -- ++(1,0) coordinate(Enode);
  \draw (Enode) to[C, l=$C_E$] (Enode |- Vee.north) node[ground]{};

\end{circuitikz}

% --- Figure 2: DC Equivalent Circuit ---
% Capacitors Open: Input/Output/Bypass removed
\begin{circuitikz}[american]
  % Power Rails
  \node[vcc](Vcc) at (2, 3) {$V_+ = \SI{15}{V}$};
  \node[vee](Vee) at (2, -3) {$V_- = \SI{-15}{V}$};

  % Transistor
  \node[npn](Q1) at (2, 0) {};

  % Resistors
  \draw (Q1.C) to[R, l=$R_C$] (Q1.C |- Vcc.south) -- (Vcc);
  \draw (Q1.E) to[R, l=$R_E$] (Q1.E |- Vee.north) -- (Vee);

  % Base Resistor (RB to Ground in this topology)
  \draw (Q1.B) -- ++(-1,0) coordinate(Bnode);
  \draw (Bnode) to[R, l=$R_B$] ++(0, -2) node[ground]{};

  % Labels for Analysis
  \draw (Q1.B) node[above left] {$V_B$};
  \draw (Q1.E) node[right] {$V_E$};
  \draw (Q1.C) node[right] {$V_C$};
\end{circuitikz}

% --- Figure 3: AC Small-Signal Model (Hybrid-pi) ---
% Capacitors Shorted: VCC/VEE -> AC Ground
\begin{circuitikz}[american]
  % Input Loop
  \draw (0,0) node[ground]{} to[sV, l=$v_{sig}$] (0,2)
  to[R, l=$R_{sig}$] (2,2) coordinate(BaseNode);

  % Base Node: RB || r_pi
  \draw (BaseNode) -- ++(0, -0.5) to[R, l=$R_B$] ++(0, -1.5) node[ground]{};
  \draw (BaseNode) -- ++(2, 0) coordinate(PiNode);

  % Hybrid-pi Transistor Model
  \draw (PiNode) to[R, l=$r_{\pi}$, v=$v_{\pi}$] (PiNode |- 0,0) node[ground]{};

  % Dependent Current Source (g_m * v_pi)
  % Gap between input and output sides
  \draw (6,0) node[ground]{} to[cI, l=$g_m v_{\pi}$] (6,2) coordinate(CollNode);

  % Output Node: RC || RL
  \draw (CollNode) -- ++(2,0) coordinate(LoadNode);
  \draw (CollNode) -- ++(0, -0.5) to[R, l=$R_C$] ++(0, -1.5) node[ground]{};
  \draw (LoadNode) to[R, l=$R_L$] (LoadNode |- 0,0) node[ground]{};

  % Output Label
  \draw (LoadNode) -- ++(1,0) node[right] {$v_o$};

\end{circuitikz}

\newpage
\section*{Lab 10: Circuit Diagrams with Design Values}

\subsection*{1. Full Experimental Schematic}
\begin{figure}[H]
    \centering
    \begin{circuitikz}[american]
        % --- Coordinates ---
        % Define the main transistor position
        \node[npn](Q1) at (6, 0) {2N3904};

        % --- Power Rails (Vertical Alignment) ---
        \draw (Q1.C) -- ++(0, 0.5) coordinate(CNode);
        \draw (CNode) to[R, l=$R_C$, a=$\SI{10.83}{k\ohm}$] ++(0, 3) node[vcc]{$V_+ = \SI{15}{V}$};
        
        \draw (Q1.E) -- ++(0, -0.5) coordinate(ENode);
        \draw (ENode) to[R, l=$R_E$, a=$\SI{14.06}{k\ohm}$] ++(0, -3) node[vee]{$V_- = \SI{-15}{V}$};

        % --- Input Stage (Left Side) ---
        % Horizontal path: Source -> Rsig -> Cc1 -> Base
        \draw (Q1.B) -- ++(-1, 0) coordinate(BaseNode); % Node for RB connection
        
        % RB to Ground (Vertical drop)
        \draw (BaseNode) to[R, l=$R_B$, a=$\SI{10}{k\ohm}$] (BaseNode |- 0, -4) node[ground]{};
        
        % Input path continuing left
        \draw (BaseNode) to[C, l=$C_{c1}$, a=$\SI{47}{\micro\farad}$] ++(-3, 0) coordinate(C1Node);
        \draw (C1Node) to[R, l=$R_{sig}$, a=$\SI{50}{\ohm}$] ++(-3, 0) coordinate(InNode);
        
        % Source to Ground
        \draw (InNode) to[sV, l=$v_{sig}$] (InNode |- 0, -4) node[ground]{};

        % --- Output Stage (Right Side) ---
        % Horizontal path: Collector -> Cc2 -> Output
        \draw (CNode) to[short, *-] ++(0,0) -- ++(1.5, 0) coordinate(C2Start);
        \draw (C2Start) to[C, l=$C_{c2}$, a=$\SI{47}{\micro\farad}$] ++(2.5, 0) coordinate(OutNode);
        
        % Output label
        \draw (OutNode) to[short, -o] ++(0.5, 0) node[right]{$v_o$};
        
        % RL to Ground (Vertical drop from Output Node)
        \draw (OutNode) to[R, l=$R_L$, a=$\SI{10}{k\ohm}$] (OutNode |- 0, -4) node[ground]{};

        % --- Emitter Bypass (Right Side) ---
        % Connection from Emitter Node -> Right -> Down through CE -> Ground
        % This matches the visual style of placing shunts vertically
        \draw (ENode) to[short, *-] ++(2, 0) coordinate(ECapNode);
        \draw (ECapNode) to[C, l=$C_E$, a=$\SI{47}{\micro\farad}$] (ECapNode |- 0, -4) node[ground]{};

    \end{circuitikz}
    \caption{Full Amplifier Schematic (Orthogonal Layout)}
\end{figure}

\subsection*{2. DC Equivalent Circuit (Bias Point)}
\begin{figure}[H]
    \centering
    \begin{circuitikz}[american]
        % Transistor centered
        \node[npn](Q1) at (4, 0) {};

        % RC and VCC
        \draw (Q1.C) -- ++(0, 0.5) coordinate(CNode);
        \draw (CNode) to[R, l=$R_C$, a=$\SI{10.83}{k\ohm}$, i<_=$I_C$] ++(0, 2.5) node[vcc]{$V_+$};
        \draw (Q1.C) node[right, xshift=0.2cm] {$V_C = \SI{4.17}{V}$};

        % RE and VEE
        \draw (Q1.E) -- ++(0, -0.5) coordinate(ENode);
        \draw (ENode) to[R, l=$R_E$, a=$\SI{14.06}{k\ohm}$, i=$I_E$] ++(0, -2.5) node[vee]{$V_-$};
        \draw (Q1.E) node[right, xshift=0.2cm] {$V_E = \SI{-0.8}{V}$};

        % RB to Ground
        \draw (Q1.B) -- ++(-1, 0) coordinate(BaseNode);
        \draw (BaseNode) to[R, l=$R_B$, a=$\SI{10}{k\ohm}$, i<_=$I_B$] (BaseNode |- 0, -3) node[ground]{};
        \draw (Q1.B) node[above left] {$V_B = \SI{-0.1}{V}$};

    \end{circuitikz}
    \caption{DC Equivalent Circuit}
\end{figure}

\subsection*{3. AC Small-Signal Model}
\begin{figure}[H]
    \centering
    \begin{circuitikz}[american]
        % --- Input Loop ---
        \draw (0,0) node[ground]{} to[sV, l=$v_{sig}$] (0,3) 
        to[R, l=$R_{sig}$, a=$\SI{50}{\ohm}$] (3,3) coordinate(BaseNode);
        
        % RB shunt
        \draw (BaseNode) to[R, l=$R_B$, a=$\SI{10}{k\ohm}$, *-] (BaseNode |- 0,0) node[ground]{};
        
        % Link to Transistor Model
        \draw (BaseNode) -- (5,3) coordinate(PiNode);
        
        % --- Hybrid-pi Model ---
        \draw (PiNode) to[R, l=$r_{\pi}$, a=$\SI{2.6}{k\ohm}$, v=$v_{\pi}$] (PiNode |- 0,0) node[ground]{};
        
        % Dependent Source
        \draw (8,0) node[ground]{} to[cI, l=$g_m v_{\pi}$, a=$(\SI{38.5}{mS})v_{\pi}$] (8,3) coordinate(CollNode);
        
        % --- Output Loop ---
        % RC shunt
        \draw (CollNode) to[short, *-] (9.5,3) coordinate(RCNode);
        \draw (RCNode) to[R, l=$R_C$, a=$\SI{10.83}{k\ohm}$] (RCNode |- 0,0) node[ground]{};
        
        % RL shunt
        \draw (RCNode) -- (12,3) coordinate(RLNode);
        \draw (RLNode) to[R, l=$R_L$, a=$\SI{10}{k\ohm}$] (RLNode |- 0,0) node[ground]{};
        
        % Output Terminal
        \draw (RLNode) to[short, -o] (13,3) node[right] {$v_o$};

    \end{circuitikz}
    \caption{Small-Signal AC Model}
\end{figure}

\end{document}
