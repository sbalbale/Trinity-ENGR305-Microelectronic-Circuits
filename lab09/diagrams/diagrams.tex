\documentclass[11pt]{article}
\usepackage[margin=1in]{geometry}
\usepackage{amsmath}
\usepackage{siunitx} % For proper unit formatting
\usepackage{booktabs} % For any tables
\usepackage{circuitikz} % Added for circuit drawings
\usepackage{float} % Added for [H] figure placement

% Set paragraph spacing instead of indentation
\setlength{\parindent}{0in}
\setlength{\parskip}{1ex}

\title{ENGR 305 Lab 9: Amplifier Schematics}
\author{Sean Balbale}
\date{\today}

\begin{document}

\maketitle

\section{Part 1: LTspice Simulation Schematic}

\begin{figure}[H]
  \centering
  \begin{circuitikz}[scale=1.2, transform shape]
    % Define power rails
    \node[vcc](Vcc) at (6, 6) {$V_+ = \SI{15}{\volt}$};
    \node[vee](Vee) at (6, -6) {$V_- = \SI{-15}{\volt}$};

    % Input Source with ground
    \draw (0,0) node[ground]{} to[sV, l={\parbox{1.3cm}{\centering $v_{sig}$ \\ \scriptsize SINE \\ \scriptsize (0 5m 1k)}}] (0,2)
    to[R, l_={\scriptsize $R_{sig}$}, l^={\scriptsize \SI{50}{\ohm}}] (2,2);

    % Coupling capacitor
    \draw (2,2) to[C, l_={\scriptsize $C_{C1}$}, l^={\scriptsize \SI{47}{\micro\farad}}] (4.5,2);

    % Gate resistor and MOSFET
    \draw (4.5,2) to[short] (4.5,3);
    \draw (4.5,2) to[R, l={\scriptsize $R_G = \SI{10}{\kilo\ohm}$}] (4.5,0) node[ground]{};
    \node[nmos, anchor=G](Q1) at (6.5,3) {};
    \draw (4.5,3) to[short, -o] (Q1.G);
    \node at (6.5,3.8) {\small 2N7000};

    % Drain circuit
    \draw (Q1.D) to[R, l={\scriptsize $R_D = \SI{5.577}{\kilo\ohm}$}] (6.5,6) -- (Vcc);
    \draw (Q1.D) to[C, l_={\scriptsize $C_{C2}$}, l^={\scriptsize \SI{47}{\micro\farad}}] (9.5,5)
    to[R, l={\scriptsize $R_L = \SI{10}{\kilo\ohm}$}] (9.5,2) node[ground]{};
    \node at (9.5,5) [right, xshift=2mm] {$v_o$};

    % Source circuit with bypass capacitor
    \draw (Q1.S) to[short] (6.5,1);
    \draw (6.5,1) to[R, l={\scriptsize $R_S = \SI{12.19}{\kilo\ohm}$}] (6.5,-2);
    \draw (6.5,-2) to[short] (6.5,-6) -- (Vee);
    \draw (6.5,1) to[C, l_={\scriptsize $C_S$}, l^={\scriptsize \SI{47}{\micro\farad}}] (8.5,1) node[ground]{};

  \end{circuitikz}
  \caption{Schematic for LTspice simulation. This circuit includes the signal source $v_{sig}$ and its internal resistance $R_{sig}$. It uses the exact calculated resistor values for $R_D$ and $R_S$ to verify the hand calculations.}
  \label{fig:ltspice_model}
\end{figure}

\newpage
\section{Part 2: Physical Build Schematic}

\begin{figure}[H]
  \centering
  \begin{circuitikz}[scale=1.2, transform shape]
    % Define power rails
    \node[vcc](Vcc) at (6, 6) {$V_+ = \SI{15}{\volt}$};
    \node[vee](Vee) at (6, -6) {$V_- = \SI{-15}{\volt}$};

    % Input from function generator
    \draw (0,2) to[short, o-] (0.5,2);
    \node at (0,2) [left, xshift=-1mm] {\scriptsize From Func. Gen.};

    % Coupling capacitor
    \draw (0.5,2) to[C, l_={\scriptsize $C_{C1}$}, l^={\scriptsize \SI{47}{\micro\farad}}] (3.5,2);

    % Gate resistor and MOSFET
    \draw (3.5,2) to[short] (3.5,3);
    \draw (3.5,2) to[R, l={\scriptsize $R_G = \SI{10}{\kilo\ohm}$}] (3.5,0) node[ground]{};
    \node[nmos, anchor=G](Q1) at (6,3) {};
    \draw (3.5,3) to[short, -o] (Q1.G);
    \node at (6,3.8) {\small 2N7000};

    % Drain circuit with output to oscilloscope
    \draw (Q1.D) to[R, l={\scriptsize $R_D = \SI{5.6}{\kilo\ohm}$}] (6,6) -- (Vcc);
    \draw (Q1.D) to[C, l_={\scriptsize $C_{C2}$}, l^={\scriptsize \SI{47}{\micro\farad}}] (9.5,5);
    \draw (9.5,5) to[short, -o] (10.5,5);
    \node at (10.5,5) [right, xshift=1mm] {\scriptsize To Oscilloscope};
    \draw (9.5,5) to[R, l={\scriptsize $R_L = \SI{10}{\kilo\ohm}$}] (9.5,2) node[ground]{};

    % Source circuit with bypass capacitor
    \draw (Q1.S) to[short] (6,1);
    \draw (6,1) to[R, l={\scriptsize $R_S = \SI{12.1}{\kilo\ohm}$}] (6,-2);
    \draw (6,-2) to[short] (6,-6) -- (Vee);
    \draw (6,1) to[C, l_={\scriptsize $C_S$}, l^={\scriptsize \SI{47}{\micro\farad}}] (8,1) node[ground]{};

  \end{circuitikz}
  \caption{Schematic for physical breadboard implementation. $R_{sig}$ is not included as it is internal to the function generator. Standard resistor values ($5.6\text{k}\Omega$, $12.1\text{k}\Omega$) are used for $R_D$ and $R_S$.}
  \label{fig:physical_model}
\end{figure}

\newpage
\section{AC Small-Signal Model Schematics}
These diagrams show the equivalent circuit used for AC analysis, after all DC sources are grounded and all capacitors are treated as short circuits.

\begin{figure}[H]
  \centering
  \begin{circuitikz}[scale=1.2, transform shape]
    % Input side
    \draw (0,0) node[ground]{} to[sV, l_={\scriptsize $v_{sig}$}] (0,2.5) to[R, l_={\scriptsize $R_{sig}$}] (2.5,2.5)
    coordinate (Vin) to[short] (4,2.5) coordinate (Vi)
    node[above, yshift=1mm] {\small $v_i$};

    % Gate resistor
    \draw (Vi) to[R, l={\scriptsize $R_G$}, *-] (4,0) node[ground](GND){};

    % Small-signal model
    \draw (Vi) -- (5.5,2.5) node[above, yshift=2mm]{\small G}; % Gate
    \draw (5.5,0) node[ground]{} node[below, yshift=-2mm]{\small S}; % Source

    % Dependent current source
    \draw (8,3) to[cI, l_={\scriptsize $g_m v_{gs}$}] (8,0) node[ground]{};

    % Input voltage v_gs
    \draw [dashed] (5.5,2.5) -- (5.5,0);
    \draw [<->] (6.5, 2.5) -- (6.5, 0) node [midway, right] {\scriptsize $v_{gs} = v_i$};

    % Output side
    \draw (8,3) -- (11,3) coordinate (Vo) node[above, yshift=1mm] {\small $v_o$};
    \draw (8,0) -- (14,0);

    % Output resistors - spaced out
    \draw (Vo) to[R, l={\scriptsize $r_o$}, *-] (11,0);
    \draw (Vo) to[short] (12.5,3) to[R, l={\scriptsize $R_D$}] (12.5,0);
    \draw (Vo) to[short] (14,3) to[R, l={\scriptsize $R_L$}] (14,0);
  \end{circuitikz}
  \caption{AC Small-Signal Model (Hybrid-$\pi$ Model) showing all parallel output resistances.}
  \label{fig:ac_model_full}
\end{figure}

\begin{figure}[H]
  \centering
  \begin{circuitikz}[scale=1.2, transform shape]
    % Input side
    \draw (0,0) node[ground]{} to[sV, l_={\scriptsize $v_{sig}$}] (0,2.5) to[R, l_={\scriptsize $R_{sig}$}] (2.5,2.5)
    coordinate (Vin) to[short] (4,2.5) coordinate (Vi)
    node[above, yshift=1mm] {\small $v_i$};

    % Gate resistor
    \draw (Vi) to[R, l={\scriptsize $R_G$}, *-] (4,0) node[ground](GND){};

    % Small-signal model (pi-model)
    \coordinate (G) at (5.5,2.5);
    \coordinate (S) at (5.5,0);
    \draw (Vi) -- (G) node[above, yshift=2mm]{\small G};
    \draw (S) node[ground]{} node[below, yshift=-2mm]{\small S};
    \draw [dashed] (G) -- (S);
    \draw [<->] (6.5, 2.5) -- (6.5, 0) node [midway, right] {\scriptsize $v_{gs} = v_i$};

    % Dependent current source
    \coordinate (D) at (8,2.5);
    \draw (D) to[cI, l_={\scriptsize $g_m v_{gs}$}] (8,0) node[ground]{};

    % Output side - Grouped parallel resistors
    \draw (D) -- (11,2.5) coordinate (Vo) node[above, yshift=1mm] {\small $v_o$};
    \draw (Vo) to[R, l={\scriptsize $R_{eq} = R_D \parallel r_o \parallel R_L$}] (11,0) node[ground]{};
  \end{circuitikz}
  \caption{Simplified AC Small-Signal Model, with all output resistances combined into $R_{eq}$.}
  \label{fig:ac_model_simple}
\end{figure}

\end{document}
