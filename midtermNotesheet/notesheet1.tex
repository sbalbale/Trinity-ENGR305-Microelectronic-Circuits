\documentclass[12pt, letterpaper]{article}
\usepackage[margin=1in]{geometry}
\usepackage{amsmath}
\usepackage{amssymb}
\usepackage{graphicx}

% Title
\title{Engineering Midterm Notesheet\\Chapters 3-5}
\author{ENGR 305}
\date{} % No date

\begin{document}

\maketitle
\thispagestyle{empty}

\subsection*{Key Constants \& Values}
\begin{itemize}
    \item \textbf{Elementary Charge ($q$):} $1.602 \times 10^{-19} \text{ C}$
    \item \textbf{Thermal Voltage ($V_T$) at 300K (Room Temp):} $V_T = kT/q = 25.9 \text{ mV}$
    \item \textbf{Intrinsic Silicon ($n_i$) at 300K:} $\approx 1.5 \times 10^{10} \text{ carriers/cm}^3$ 
    \item \textbf{Permittivity of Silicon ($\epsilon_s$):} $1.04 \times 10^{-12} \text{ F/cm}$
    \item \textbf{Constant Voltage Drop (CVD) Model:} Assume $V_D \approx 0.7 \text{ V}$ for a conducting silicon diode unless specified otherwise.
\end{itemize}

\hrulefill

\section*{Chapter 3: Semiconductors}

\begin{itemize}
    \item \textbf{Intrinsic Semiconductors:} Pure silicon ($Si$). Covalent bonds. At 0K, it's an insulator. At room temp, thermal generation creates free electrons and holes.
    \begin{itemize}
        \item $n = p = n_i$ (electron and hole concentrations are equal)
        \item \textbf{Mass Action Law:} $np = n_i^2$ (holds for intrinsic and doped silicon in thermal equilibrium)
    \end{itemize}
    
    \item \textbf{Doped Semiconductors:}
    \begin{itemize}
        \item \textbf{n-type:} Doped with a pentavalent element (e.g., Phosphorus). These are \textbf{donors} ($N_D$).
        \begin{itemize}
            \item Majority carriers: electrons ($n_n \approx N_D$)
            \item Minority carriers: holes ($p_n = n_i^2 / N_D$)
        \end{itemize}
        \item \textbf{p-type:} Doped with a trivalent element (e.g., Boron). These are \textbf{acceptors} ($N_A$).
        \begin{itemize}
            \item Majority carriers: holes ($p_p \approx N_A$)
            \item Minority carriers: electrons ($n_p = n_i^2 / N_A$)
        \end{itemize}
    \end{itemize}
    
    \item \textbf{Current Flow Mechanisms:}
    \begin{itemize}
        \item \textbf{Drift Current:} Movement of carriers due to an electric field ($E$).
        \begin{itemize}
            \item Electron drift velocity: $v_{n-drift} = -\mu_n E$
            \item Hole drift velocity: $v_{p-drift} = \mu_p E$
            \item Total Drift Current Density ($J_{drift}$): $J_{drift} = q(n\mu_n + p\mu_p)E$
            \item ($\mu_n$ and $\mu_p$ are electron and hole mobilities)
        \end{itemize}
        \item \textbf{Diffusion Current:} Movement of carriers from a high-concentration area to a low-concentration area (due to a concentration gradient).
        \begin{itemize}
            \item Hole Diffusion: $J_p = -qD_p \frac{dp}{dx}$
            \item Electron Diffusion: $J_n = qD_n \frac{dn}{dx}$
            \item ($D_n$ and $D_p$ are the diffusion constants)
        \end{itemize}
    \end{itemize}
    
    \item \textbf{Einstein Relation:} Relates mobility ($\mu$) and diffusion constant ($D$).
    \begin{itemize}
        \item $D_n = V_T \mu_n$
        \item $D_p = V_T \mu_p$
    \end{itemize}
\end{itemize}

\hrulefill

\section*{Chapter 4: Diodes}

\begin{itemize}
    \item \textbf{The $pn$ Junction (Open Circuit):}
    \begin{itemize}
        \item A \textbf{depletion region} (or space-charge region) forms at the junction, clear of free carriers.
        \item Contains uncovered bound charges: positive donors ($N_D$) on the n-side, negative acceptors ($N_A$) on the p-side.
        \item This charge creates a built-in voltage ($V_0$): $V_0 = V_T \ln\left(\frac{N_A N_D}{n_i^2}\right)$.
        \item Two currents balance at equilibrium: \textbf{diffusion current} ($I_D$) and \textbf{drift current} ($I_S$). $I_S$ is the saturation current.
    \end{itemize}
    
    \item \textbf{Diode I-V Models:}
    \begin{enumerate}
        \item \textbf{Exponential Model (Shockley Equation):} Most accurate.
        \begin{itemize}
            \item $i_D = I_S (e^{v_D / (nV_T)} - 1)$
            \item $I_S$ is the saturation current (typically $10^{-15}$ A).
            \item $n$ is the ideality factor (usually 1 to 2).
            \item At room temp, $V_T \approx 25 \text{ mV}$.
            \item For $v_D \gg V_T$, $i_D \approx I_S e^{v_D / (nV_T)}$.
            \item A 60mV increase in $v_D$ (for $n=1$) increases $i_D$ by $\approx 10\times$.
        \end{itemize}
        \item \textbf{Constant-Voltage-Drop (CVD) Model:} Most common for hand analysis.
        \begin{itemize}
            \item If conducting: $v_D = 0.7 \text{ V}$ (acts as a 0.7V battery).
            \item If off: $i_D = 0$ (acts as an open circuit).
            \item Turn-on threshold is $v_D \approx 0.5 \text{ V}$ (cut-in voltage).
        \end{itemize}
        \item \textbf{Ideal Diode Model:} A simple switch.
        \begin{itemize}
            \item If conducting (forward-biased): $v_D = 0 \text{ V}$ (short circuit).
            \item If off (reverse-biased): $i_D = 0$ (open circuit).
        \end{itemize}
    \end{enumerate}
    
    \item \textbf{Small-Signal Model (for DC bias point Q):}
    \begin{itemize}
        \item First, solve for DC values ($I_D, V_D$) using the CVD or exponential model.
        \item Then, for small ac signals, replace the diode with its \textbf{small-signal resistance ($r_d$)}:
        \item $r_d = \frac{nV_T}{I_D}$
    \end{itemize}
    
    \item \textbf{Zener Diode (Breakdown Region):}
    \begin{itemize}
        \item Used for voltage regulation. Operates in reverse breakdown.
        \item Model: $V_Z = V_{Z0} + r_z I_Z$
        \begin{itemize}
            \item $V_{Z0}$: Zener voltage at $I_Z \approx 0$.
            \item $r_z$: Zener incremental resistance.
        \end{itemize}
    \end{itemize}
    
    \item \textbf{Diode Capacitance:}
    \begin{itemize}
        \item \textbf{Junction/Depletion Capacitance ($C_j$):} Dominant in reverse bias.
        \begin{itemize}
            \item $C_j = \frac{C_{j0}}{(1 + V_R/V_0)^m}$ (where $m \approx 0.5$ for abrupt junction)
        \end{itemize}
        \item \textbf{Diffusion Capacitance ($C_d$):} Dominant in forward bias. Proportional to current.
        \begin{itemize}
            \item $C_d = \left(\frac{\tau_T}{V_T}\right) I_D$ (where $\tau_T$ is transit time)
        \end{itemize}
    \end{itemize}
    
    \item \textbf{Rectifier Circuits:}
    \begin{itemize}
        \item \textbf{Half-Wave (HW):} One diode. PIV = $V_s(\text{peak})$. $V_O(\text{peak}) = V_s(\text{peak}) - V_D$.
        \item \textbf{Full-Wave (FW) - Center-Tapped:} Two diodes. PIV = $2V_s(\text{config}) - V_D$. $V_O(\text{peak}) = V_s(\text{peak}) - V_D$.
        \item \textbf{Full-Wave (FW) - Bridge:} Four diodes. PIV = $V_s(\text{peak}) - V_D$. $V_O(\text{peak}) = V_s(\text{peak}) - 2V_D$.
    \end{itemize}
    
    \item \textbf{Peak Rectifier (HW with Filter Capacitor C):}
    \begin{itemize}
        \item Output $V_O$ is approximately the peak input, $V_p$, minus a small \textbf{ripple voltage ($V_r$)}.
        \item Capacitor discharges through $R_L$ when diode is off.
        \item For $CR_L \gg T$ (period of input): $V_r \approx \frac{V_p}{fCR_L} = \frac{I_L}{fC}$
        \item For FW Rectifier: $V_r \approx \frac{V_p}{2fCR_L} = \frac{I_L}{2fC}$
        \item Diode conducts for a short interval ($\Delta t$) near the peak to replenish capacitor charge.
        \item Diode current is large and pulsed: $i_{D,avg} \approx I_L(1 + \pi\sqrt{2V_p/V_r})$.
    \end{itemize}
\end{itemize}

\hrulefill

\section*{Chapter 5: MOSFETs}

\begin{itemize}
    \item \textbf{Structure:} n-channel (NMOS) and p-channel (PMOS). Terminals are Gate (G), Drain (D), Source (S), Body (B).
    \item \textbf{Parameters:}
    \begin{itemize}
        \item \textbf{Threshold Voltage ($V_t$):} $V_{tn}$ for NMOS (positive), $V_{tp}$ for PMOS (negative).
        \item \textbf{Oxide Capacitance ($C_{ox}$):} $C_{ox} = \epsilon_{ox} / t_{ox}$.
        \item \textbf{Process Transconductance ($k'$):} $k_n' = \mu_n C_{ox}$, $k_p' = \mu_p C_{ox}$.
        \item \textbf{Transistor Transconductance ($k$):} $k_n = k_n'(W/L)$, $k_p = k_p'(W/L)$.
    \end{itemize}
    \item \textbf{Overdrive Voltage ($V_{OV}$):} The key control voltage.
    \begin{itemize}
        \item NMOS: $V_{OV} = V_{GS} - V_{tn}$
        \item PMOS: $|V_{OV}| = V_{SG} - |V_{tp}|$
    \end{itemize}
    \item \textbf{Channel-Length Modulation ($\lambda$):}
    \begin{itemize}
        \item Models the slight increase in $i_D$ with $v_{DS}$ in saturation.
        \item $i_D' = i_D (1 + \lambda v_{DS})$
        \item \textbf{Output Resistance ($r_o$):} $r_o = \frac{V_A}{I_D} \approx \frac{1}{\lambda I_D}$ (where $V_A = 1/\lambda$ is the Early Voltage).
    \end{itemize}
\end{itemize}

\subsection*{Regions of Operation (Tables 5.1 \& 5.2 are provided on exam)}

\subsubsection*{NMOS Transistor}
\begin{enumerate}
    \item \textbf{Cutoff:}
    \begin{itemize}
        \item \textbf{Condition:} $V_{GS} < V_{tn}$
        \item \textbf{Current:} $i_D = 0$
    \end{itemize}
    
    \item \textbf{Triode (Linear) Region:}
    \begin{itemize}
        \item \textbf{Conditions:} $V_{GS} \ge V_{tn}$ AND $V_{DS} < V_{OV}$ (or $V_{GD} > V_{tn}$)
        \item \textbf{Current:} $i_D = k_n \left[ (V_{GS} - V_{tn})V_{DS} - \frac{1}{2}V_{DS}^2 \right]$
        \item Acts like a voltage-controlled resistor. For small $V_{DS}$: $r_{DS} = \frac{1}{k_n V_{OV}}$.
    \end{itemize}
    
    \item \textbf{Saturation Region:} (Used for amplifiers)
    \begin{itemize}
        \item \textbf{Conditions:} $V_{GS} \ge V_{tn}$ AND $V_{DS} \ge V_{OV}$ (or $V_{GD} \le V_{tn}$)
        \item \textbf{Current:} $i_D = \frac{1}{2} k_n (V_{GS} - V_{tn})^2 = \frac{1}{2} k_n V_{OV}^2$
        \item Acts like a voltage-controlled current source.
    \end{itemize}
\end{enumerate}

\subsubsection*{PMOS Transistor} (Use $V_{SG}$, $V_{SD}$, $|V_{tp}|$, $k_p$)
\begin{enumerate}
    \item \textbf{Cutoff:}
    \begin{itemize}
        \item \textbf{Condition:} $V_{SG} < |V_{tp}|$
        \item \textbf{Current:} $i_D = 0$
    \end{itemize}
    
    \item \textbf{Triode (Linear) Region:}
    \begin{itemize}
        \item \textbf{Conditions:} $V_{SG} \ge |V_{tp}|$ AND $V_{SD} < |V_{OV}|$
        \item \textbf{Current:} $i_D = k_p \left[ (V_{SG} - |V_{tp}|)V_{SD} - \frac{1}{2}V_{SD}^2 \right]$
    \end{itemize}
    
    \item \textbf{Saturation Region:}
    \begin{itemize}
        \item \textbf{Conditions:} $V_{SG} \ge |V_{tp}|$ AND $V_{SD} \ge |V_{OV}|$
        \item \textbf{Current:} $i_D = \frac{1}{2} k_p (V_{SG} - |V_{tp}|)^2 = \frac{1}{2} k_p |V_{OV}|^2$
    \end{itemize}
\end{enumerate}

\hrulefill

\section*{Homeworks 1-4: Problem-Solving Steps}

\begin{itemize}
    \item \textbf{DC Analysis of Diode Circuits:}
    \begin{enumerate}
        \item Assume a state for each diode (ON or OFF) based on inspection.
        \item Model ON diodes as 0.7V sources (CVD model) and OFF diodes as open circuits.
        \item Solve the resulting linear circuit for all currents and voltages.
        \item \textbf{Check assumptions:}
        \begin{itemize}
            \item If a diode was assumed ON, check if $i_D > 0$.
            \item If a diode was assumed OFF, check if $v_D < 0.7 \text{ V}$.
        \end{itemize}
        \item If any assumption is wrong, change the state of that diode and re-solve.
    \end{enumerate}
    
    \item \textbf{DC Analysis of MOSFET Circuits (from Slides, e.g., Ex. 5.3, 5.6):}
    \begin{enumerate}
        \item \textbf{Assume} a region of operation for the MOSFET (usually \textbf{Saturation} for amplifiers).
        \item Write the $i_D$ equation for that region (e.g., $i_D = \frac{1}{2} k_n V_{OV}^2$).
        \item Write KVL/KCL equations for the rest of the circuit (e.g., $V_G = ...$, $V_S = ...$, $V_D = ...$).
        \item Substitute known relationships (e.g., $V_{OV} = V_{GS} - V_{tn}$, $V_{GS} = V_G - V_S$, $i_D = i_S$).
        \item Solve the system of equations for the unknown currents and node voltages (e.g., $i_D$, $V_S$, $V_D$).
        \item \textbf{Check assumptions:}
        \begin{itemize}
            \item If Saturation was assumed: Verify that $V_{DS} \ge V_{OV}$ (for NMOS) or $V_{SD} \ge |V_{OV}|$ (for PMOS).
            \item If Triode was assumed: Verify that $V_{DS} < V_{OV}$ (for NMOS) or $V_{SD} < |V_{OV}|$ (for PMOS).
        \end{itemize}
        \item If the assumption is wrong, re-assume the correct region and re-solve from Step 2.
    \end{enumerate}
    
    \item \textbf{Small-Signal Diode Problems:}
    \begin{enumerate}
        \item First, solve the DC circuit (using CVD model) to find the DC bias current $I_D$.
        \item Calculate the small-signal resistance: $r_d = nV_T / I_D$.
        \item Create the small-signal equivalent circuit: Replace the diode with $r_d$, turn off DC voltage sources (short to ground), and turn off DC current sources (open circuit).
        \item Solve the small-signal circuit for the required ac quantity (e.g., $v_o / v_i$).
    \end{enumerate}
\end{itemize}

\end{document}